\documentclass[11pt, uplatex, dvipdfmx]{jsarticle} 
\usepackage{amsmath,amsthm,amssymb,amsfonts,fancyhdr, enumerate, braket, setspace, graphicx, bm}
\newcommand{\ds}{\displaystyle}

\theoremstyle{definition}
\newtheorem{theorem}{定理}[section]
\newtheorem*{theorem*}{定理}
\newtheorem*{definition}{定義}
\newtheorem{lemma}{補題}
\newtheorem*{remark}{注意}
\newtheorem{example}{例}
\newtheorem*{notation}{記号}
\renewcommand{\proofname}{\textbf{証明}}


\renewcommand{\arraystretch}{1.2}

\pagestyle{plain}


\title{Gram-Schmidt の正規直交化法について}
\date{}


\begin{document}

\maketitle

\begin{abstract}
  Gram-Schmidt の正規直交化法 (GSO) は内積空間の与えられた基底から正規
  直交基底を構成する方法である.本稿では GSO により得られる直交基底およ
  び正規直交基底の明示公式を与える.さらに,直交基底への変換行列を Gram
  行列から行基本変形によって計算する方法についても述べる.
\end{abstract}

\section{はじめに}

本稿では有限次元実内積空間のみを扱い,記号・記法は概ね\cite{miyake}の流
儀に従う.

実内積空間 $V$ の基底 $\{\bm{v}_1, \ldots, \bm{v}_n\}$ に対して
\[
  \bm{w}_1 = \bm{v}_1, \quad \bm{w}_k = \bm{v}_k - \sum_{i=1}^{k-1}
  \frac{(\bm{v}_k, \bm{w}_i)}{(\bm{w}_i, \bm{w}_i)} \bm{w}_i \; (2 \leq k
  \leq n)
\]
とすると $\{\bm{w}_1, \ldots, \bm{w}_n\}$ は $V$ の直交基底であり,さら
に,各 $k=1, \ldots, n$ に対して
\[
  \bm{u}_k = \frac{\bm{w}_k}{\|\bm{w}_k\|}
\]
とすれば $\{\bm{u}_1, \ldots, \bm{u}_n\}$ は $V$ の正規直交基底となる,
というのが Gram-Schmidt の正規直交化法 (GSO) のアルゴリズムである.直交
基底 $\{\bm{w}_1, \ldots, \bm{w}_n\}$ も正規直交基底
$\{\bm{u}_1, \ldots, \bm{u}_n\}$ もはじめに与えられた基
底 $\{\bm{v}_1, \ldots, \bm{v}_n\}$ の1次結合で表せるので,本稿ではその
係数たちを明示する公式(定理\ref{thm:formula})を与える.特に,
\[
  (\bm{w}_1, \ldots, \bm{w}_n) = (\bm{v}_1, \ldots, \bm{v}_n) M, \quad
  (\bm{u}_1, \ldots, \bm{u}_n) = (\bm{w}_1, \ldots, \bm{w}_n)N
\]
となる基底の変換行列 $M, N $ を Gram 行列 $G=\left[(\bm{v}_i,
  \bm{v}_j)\right]$ を用いて明示する.また,Gram 行列 $G$
からの行基本変形によって変換行列 $M$ が求められることも紹介する.


\begin{notation}
  正方行列 $G$ の $k$ 次主対角行列を $G^{(k)}$ と書き,その $(i,j)$ 余
  因子を $\tilde{g}^{(k)}_{ij}$ と書く.つまり,
  \[
    G^{(k)} := \left[
      \begin{array}{ccc}
        g_{11} & \cdots & g_{1k}\\
        \vdots & \ddots & \vdots\\
        g_{k1} & \cdots & g_{kk}
      \end{array}
    \right], \quad \tilde{g}^{(k)}_{ij} := (-1)^{i+j} \left|G^{(k)}_{ij}\right|
  \]
  とする.ここで,$G^{(k)}_{ij}$ は $G^{(k)}$ から $i$ 行と $j$ 列を取
  り除いた $k-1$ 次正方行列を表す.
\end{notation}

\section{GSO による直交基底,正規直交基底の明示公式}

実内積空間 $V$ の基底 $\{\bm{v}_1, \ldots, \bm{v}_n\}$ に対し,基底同士
の内積からなる行列 $G=\left[ (\bm{v}_i, \bm{v}_j)\right]$ を Gram 行列
という.$G$ は正定値対称行列である.つまり,${}^{t}G =
G$ かつ $\bm{0}$ でない全ての $\bm{x} \in \mathbb{R}^n$ に対し
て ${}^{t}\bm{x} G \bm{x} >0$ である.Gram 行列は内積の計算を
\[
  \left( \sum_{i=1}^{n} x_i \bm{v}_i,\;  \sum_{i=1}^{n} y_i
    \bm{v}_i\right)  = {}^{t}\bm{x} G \bm{y} \quad \left( \bm{x} =\left[
    \begin{array}{c}
      x_1\\
      \vdots\\
      x_n
    \end{array}
  \right], \; \bm{y} = \left[
    \begin{array}{c}
      y_1\\
      \vdots\\
      y_n
    \end{array}
  \right]\right)
\]
と行列の積に帰着させる.GSO によって得られる直交基底や正規直交基底は,
はじめに与えられた基底 $\{\bm{v}_1, \ldots, \bm{v}_n\}$ の1次結合で書け
るので,その係数たちを $G$ を用いて明示する.



\begin{theorem}\label{thm:formula}
  $\{\bm{v}_1, \ldots, \bm{v}_n\}$ を実内積空間 $V$ の基底とし,そ
  の Gram 行列を $G=\left[ g_{ij}\right]$ とする.
  \[
    \bm{w}_1= \bm{v}_1, \quad \bm{w}_k =
    \frac{1}{\left|G^{(k-1)}\right|}\left|
      \begin{array}{ccc}
        g_{11} & \cdots & g_{1k}\\
        \vdots & \ddots & \vdots\\
        g_{k-1,1} & \cdots & g_{k-1, k}\\
        \bm{v}_1 & \cdots & \bm{v}_k
      \end{array}
    \right| =  \frac{1}{\left|G^{(k-1)}\right|} \sum_{i=1}^{k}
    \tilde{g}^{(k)}_{ik} \bm{v}_i \quad ( 2 \leq k \leq n)
  \]
  とすれば,$\{\bm{w}_1, \ldots, \bm{w}_n\}$ は GSO によって得られる $V$ の直交基底である.さらに,
  \[
    (\bm{w}_1, \bm{w}_1) = g_{11}, \quad (\bm{w}_k, \bm{w}_k) =
    \frac{\left| G^{(k)}\right|}{\left|G^{(k-1)}\right|} \; (2 \leq k \leq n)
  \]
  となる.これより
  \[
    \bm{u}_1 = \frac{\bm{w}_1}{\sqrt{g_{11}}}, \quad \bm{u}_k =
    \sqrt{\frac{\left|G^{(k-1)}\right|}{\left|G^{(k)}\right|}}~
    \bm{w}_k \; (2 \leq k \leq n)
  \]
  とすれば,$\{\bm{u}_1, \ldots, \bm{u}_n\}$ は
  $\{\bm{w}_1, \ldots, \bm{w}_n\}$ を正規化した $V$ の正規直交基底であ
  る.つまり,
  \[
    \mu_{ij} = \frac{\tilde{g}^{(j)}_{ij}}{\tilde{g}^{(j)}_{jj}}
    = \frac{\tilde{g}^{(j)}_{ij}}{\left|G^{(j-1)}\right|}, \quad
    \nu_{i} = \sqrt{\frac{\left|G^{(i-1)}\right|}{\left|G^{(i)}\right|}}
  \]
  とおいて
  \[
    M = \left[
      \begin{array}{ccccc}
        1 & \mu_{12} & \mu_{13} & \cdots & \mu_{1n}\\
         & 1 & \mu_{23} & \cdots & \mu_{2n}\\
         &  & 1 & \cdots & \mu_{3n}\\
         &  &  & \ddots & \vdots\\
        O &  &  &  & 1
      \end{array}
    \right], \quad N= \left[
      \begin{array}{ccccc}
        \sqrt{1/g_{11}} & & & & O\\
                        & \nu_{2} & & & \\
                        & & \nu_{3} & &\\
                        & && \ddots & \\
        O & & &&  \nu_{n}
      \end{array}
    \right]
  \]
  とすれば,以下が成り立つ.
  \[
    \begin{aligned}
      \left(\bm{u}_1, \bm{u}_2, \cdots, \bm{u}_n\right) = \left(\bm{w}_1, \bm{w}_2, \cdots, \bm{w}_n\right) N=
    \left( \bm{v}_1 , \bm{v}_2, \cdots, \bm{v}_n\right)MN
  \end{aligned}
  \]
\end{theorem}

\begin{proof}
  $\{\bm{v}_1, \ldots, \bm{v}_n\}$ から GSO により得られる直交基底
  を $\{\bm{w}_1, \ldots, \bm{w}_n\}$ とする.つまり,
  \[
    \bm{w}_1 = \bm{v}_1, \quad \bm{w}_k = \bm{v}_k - \sum_{i=1}^{k-1}
    \frac{(\bm{v}_k, \bm{w}_{i})}{(\bm{w}_i,\bm{w}_i)}\bm{w}_i \; (2 \leq k \leq n)
  \]
  とする.このとき,各 $k=2,\ldots, n$ に対して
  \[
    \langle \bm{w}_1, \ldots, \bm{w}_{k-1}\rangle =\langle \bm{v}_1,
    \ldots, \bm{v}_{k-1}\rangle
  \]
  なので,
  \[
    \bm{w}_k = \sum_{i=1}^{k-1}\mu_{i k} \bm{v}_i + \bm{v}_k = \left(\bm{v}_1, \ldots, \bm{v}_{k}\right) \left[
      \begin{array}{c}
        \mu_{1k}\\
        \vdots\\
        \mu_{k-1,k}\\
        1
      \end{array}
    \right]\; (\mu_{ik} \in \mathbb{R})
  \]
  と書ける(添字の片方または両方が2文字以上になるときはコンマで区切る).つ
  まり,
  \[
    M = \left[
      \begin{array}{ccccc}
        1 & \mu_{12} & \mu_{13} & \cdots & \mu_{1n}\\
          & 1 & \mu_{23} & \cdots & \mu_{2n}\\
          &  & 1 & \cdots & \mu_{3n}\\
          &  &  & \ddots & \vdots\\
        O &  &  &  & 1
      \end{array}
    \right]
  \]
  によって $(\bm{w}_1, \ldots, \bm{w}_n) = (\bm{v}_1, \ldots,
  \bm{v}_n) M$ と書ける.この各 $\mu_{ij}$ を決定する.
  
  各 $k=2,\ldots, n$ に対して
  $\bm{w}_k \in \langle \bm{w}_1, \ldots, \bm{w}_{k-1}\rangle^{\perp} =
  \langle\bm{v}_1, \ldots, \bm{v}_{k-1}\rangle^{\perp} $ なので,
  \[
    (\bm{v}_1, \bm{w}_k) = (\bm{v}_2, \bm{w}_k) = \cdots = (\bm{v}_{k-1}, \bm{w}_k) = 0
  \]
  である.従って,$\{\bm{e}^k_{1}, \ldots,
  \bm{e}^k_k\}$ を $\mathbb{R}^k$ の標準基底とすれば,各 $i=1,
  \ldots, k-1$ に対して
  \[
    0  =(\bm{v}_i, \bm{w}_k) = {}^{t}\bm{e}^k_{i}
    G^{(k)} \left[
      \begin{array}{c}
        \mu_{1k}\\
        \vdots\\
            1
      \end{array}
    \right]
    = \left[
      \begin{array}{ccc}
        g_{i1} & \cdots & g_{ik}
      \end{array}
    \right] \left[
      \begin{array}{c}
        \mu_{1k}\\
        \vdots \\
        1
      \end{array}
    \right] 
  \]
  である.従って,$\bm{g}_i = {}^{t}\left[
    \begin{array}{ccc}
      g_{i1} & \cdots & g_{ik}
    \end{array}
  \right], \; \bm{\mu}_k = {}^{t}\left[
    \begin{array}{cccc}
      \mu_{1k} & \cdots & \mu_{k-1,k} & 1
    \end{array}
  \right]$ とすれば,$\bm{\mu}_k$ は $\mathbb{R}^k$ の標準内積のもと
  で $\bm{g}_1, \ldots, \bm{g}_{k-1}$ に直交するので以下のように書ける.
  \[
    \bm{\mu}_k = m_k \left|
      \begin{array}{ccc}
        g_{11} & \cdots & g_{1k}\\
        \vdots & \ddots & \vdots\\
        g_{k-1,1} & \cdots & g_{k-1,k}\\
        \bm{e}_1 & \cdots & \bm{e}_k
      \end{array}
    \right| = m_k \sum_{j=1}^{k} \tilde{g}^{(k)}_{jk} \bm{e}^k_j
    = m_k \left[
      \begin{array}{c}
        \tilde{g}^{(k)}_{1k}\\
        \vdots\\
        \tilde{g}^{(k)}_{kk}
      \end{array}
    \right]\quad ( m_k \in \mathbb{R})
  \]
  $\bm{\mu}_k$ の $\bm{e}^k_k$ 成分は $1$ なので
  $m_k = 1/\tilde{g}^{(k)}_{kk} = 1/\left|G^{(k-1)}\right|$ である.以
  上から,$k=2, \ldots, n$ に対して
  \[
    \bm{w}_k = (\bm{v}_1, \ldots, \bm{v}_k) \bm{\mu}_k
    = \frac{1}{\left| G^{(k-1)} \right|} \sum_{i=1}^{k} \tilde{g}^{(k)}_{ik} \bm{v}_k
    = \frac{1}{\left|G^{(k-1)} \right|}\left|
      \begin{array}{ccc}
        g_{11} & \cdots & g_{1k}\\
        \vdots & \ddots & \vdots\\
        g_{k-1,1} & \cdots & g_{k-1,k}\\
        \bm{v}_1 & \cdots & \bm{v}_k
      \end{array}
    \right|
  \]
  を得る.さらに,$\|\bm{w}_1\| = \sqrt{g_{11}}$ であり,$k=2, \ldots, n$ に対しては
  \[
    \begin{aligned}
      (\bm{w}_k, \bm{w}_k) &= {}^{t}\bm{\mu}_k G^{(k)} \bm{\mu}_k = {}^{t} \bm{\mu}_k \left[
        \begin{array}{c}
          {}^{t}\bm{e}^k_1 G^{(k)}\bm{\mu}_k\\
          \vdots \\
          {}^{t} \bm{e}^k_{k} G^{(k)}\bm{\mu}_k\\
        \end{array}
      \right] = \left[
        \begin{array}{cccc}
          \mu_{1k} & \cdots & \mu_{k-1,k} & 1
        \end{array}
      \right] \left[
        \begin{array}{c}
          0\\
          \vdots\\
          0\\
          {}^{t}\bm{e}^k_k G^{(k)}\bm{\mu}_k
        \end{array}
      \right]\\
      &= {}^{t} \bm{e}^k_k G^{(k)}\bm{\mu}_k = \sum_{i=1}^{k-1} g_{ki} \mu_{ik} + g_{kk}
      = \frac{1}{\left| G^{(k-1)}\right|}\sum_{i=1}^{k}g_{ik} ~ \tilde{g}^{(k)}_{ik}
      = \frac{\left| G^{(k)}\right|}{\left|G^{(k-1)}\right|}
    \end{aligned}
  \]
  なので,$\nu_k = 1/\|\bm{w}_k\| = \sqrt{\left|G^{(k-1)}\right|/\left|G^{(k)}\right|}$ として
  \[
    N = \left[
      \begin{array}{cccc}
        1/\sqrt{g_{11}} & & & O\\
                       & \nu_2 & & \\
                       & & \ddots & \\
        O & & & \nu_n
      \end{array}
    \right]
  \]
  によって $(\bm{u}_1, \ldots, \bm{u}_m) = (\bm{w}_1, \ldots, \bm{w}_n) N$ と書ける.
\end{proof}


\begin{remark}
  各 $\bm{w}_k \; (k \geq 2)$ は $\bm{v}_1, \ldots, \bm{v}_{k-1}$ に直
  交するので,
  \[
    \begin{aligned}
      \bm{0}_{k-1} &= \left[
        \begin{array}{c}
          (\bm{v}_1, \bm{w}_k)\\
          \vdots\\
          (\bm{v}_{k-1}, \bm{w}_k)
        \end{array}
      \right] = \left[
        \begin{array}{c}
          {}^{t}\bm{e}_1 G^{(k)} \bm{\mu_k}\\
          \vdots\\
          {}^{t} \bm{e}_{k-1} G^{(k)} \bm{\mu}_k
        \end{array}
      \right] = \left[
        \begin{array}{c}
          {}^{t}\bm{e}_1 G^{(k)}\\
          \vdots\\
          {}^{t}\bm{e}_{k-1}G^{(k)}
        \end{array}
      \right] \bm{\mu}_k\\
      &=\left[
        \begin{array}{cc}
          & g_{1k}\\
          G^{(k-1)} & \vdots \\
           & g_{k-1,k}
        \end{array}
      \right] \left[
        \begin{array}{c}
          \mu_{1k}\\
          \vdots\\
          \mu_{k-1,k}\\
          1
        \end{array}
      \right]
      = G^{(k-1)} \left[
        \begin{array}{c}
          \mu_{1k}\\
          \vdots \\
          \mu_{k-1,k}
        \end{array}
        \right] + \left[
          \begin{array}{c}
            g_{1k}\\
            \vdots\\
            g_{k-1,k}
          \end{array}
        \right]
    \end{aligned}
  \]
  となる.つまり,$\left[
    \begin{array}{c}
      \mu_{1k}\\
      \vdots\\
      \mu_{k-1,k}
    \end{array}
    \right]$ は連立1次方程式 $G^{(k-1)} \bm{x} = -\left[
      \begin{array}{c}
        g_{1k}\\
        \vdots\\
        g_{k-1,k}
      \end{array}
\right]$ の解でもある.
\end{remark}

\begin{example}\label{exmp:3-dim}
  $3$ 次元実内積空間 $V$ の基底 $\{\bm{v}_1, \bm{v}_2,
  \bm{v}_3\}$ の Gram 行列が以下のように与えられるとする.
  \[
    G= \left[(\bm{v}_i, \bm{v}_j\right] = \left[
      \begin{array}{rrr}
        10 & 4 & 3\\
        4 & 2 & 1\\
        3 & 1 & 5
      \end{array}
    \right]
  \]
  定理\ref{thm:formula}から,GSO により得られる直交基底
  $\{\bm{w}_1, \bm{w}_2, \bm{w}_3\}$ は以下のように計算できる.
  \[
    \begin{aligned}
      \bm{w}_1 &=\bm{v}_1\\
      \bm{w}_2 &= \frac{1}{10}\left|
        \begin{array}{cc}
          10  & 4\\
          \bm{v}_1 & \bm{v}_2
        \end{array}
      \right| = \frac{1}{10}\left( -4\bm{v}_1 + 10 \bm{v}_2\right) =
      -\frac{2}{5}\bm{v}_1 + \bm{v}_2\\
      \bm{w}_3 &= \frac{1}{\left|
          \begin{array}{rr}
            10 & 4\\
            4 & 2
          \end{array}
        \right|} \left|
        \begin{array}{rrr}
          10 & 4 & 3\\
          4 & 2 & 1\\
          \bm{v}_1 & \bm{v}_2 & \bm{v}_3
        \end{array}
      \right| = \frac{1}{4} \left( \left|
          \begin{array}{rr}
            4 & 3\\
            2 & 1
          \end{array}
        \right| \bm{v}_1 - \left|
          \begin{array}{rr}
            10 & 3\\
            4 & 1
          \end{array}
        \right|\bm{v}_2 \right) + \bm{v}_3\\
      &= -\frac{1}{2}\bm{v}_1 + \frac{1}{2}\bm{v}_2 + \bm{v}_3
    \end{aligned}
  \]
  あるいは,その後の注意で述べたように方程式 $\left[
      \begin{array}{rr}
        10 & 4\\
        4 & 2
      \end{array}
    \right] \left[
      \begin{array}{c}
        \mu_{13}\\
        \mu_{23}
      \end{array}
    \right]= -\left[
      \begin{array}{r}
        3\\
        1
      \end{array}
    \right]$ を解いて
  \[
    \bm{w}_3 = \mu_{13} \bm{v}_1 + \mu_{23} \bm{v}_2 + \bm{v}_3 =
    -\frac{1}{2}\bm{v}_1 + \frac{1}{2}\bm{v}_2 + \bm{v}_3
  \]
  と計算することもできる.さらに,それぞれの自己内積は
  \[
    (\bm{w}_1, \bm{w}_2) = 10, \quad (\bm{w}_2, \bm{w}_2) = \frac{\left|
        \begin{array}{rr}
          10 & 4\\
          4 & 2
        \end{array}
      \right|}{10} = \frac{2}{5}, \quad (\bm{w}_3, \bm{w}_3) = \frac{\left|
        \begin{array}{rrr}
          10 & 4 & 3\\
          4 & 2 & 1\\
          3 & 1 & 5
        \end{array}
      \right|}{\left|
        \begin{array}{rr}
          10 & 4\\
          4 & 2
        \end{array}
      \right|} =4
  \]
  である.従って,GSO で得られる $V$ の正規直交基底 $\{\bm{u}_1, \bm{u}_2, \bm{u}_3\}$ は
  \[
    \bm{u}_1 = \frac{1}{\sqrt{10}}~\bm{w}_1, \quad \bm{u}_2 =
    \frac{\sqrt{10}}{2}~\bm{w}_2 , \quad \bm{u}_3 = \frac{1}{2}~\bm{w}_3
  \]
  である.つまり,基底間の変換は以下のように書ける.
  \[
    (\bm{w}_1, \bm{w}_2, \bm{w}_3) = (\bm{v}_1, \bm{v}_2, \bm{v}_3) \left[
      \begin{array}{rrr}
        1 & -\frac{2}{5} & -\frac{1}{2}\\
        0 & 1 & \frac{1}{2}\\
        0 & 0 & 1
      \end{array}
    \right], \quad (\bm{u}_1, \bm{u}_2, \bm{u}_3) = (\bm{w}_1, \bm{w}_2, \bm{w}_3)
    \left[
      \begin{array}{rrr}
        \frac{1}{\sqrt{10}} & 0 & 0\\
        0 & \frac{\sqrt{10}}{2} & 0\\
        0 & 0 & \frac{1}{2}
      \end{array}
    \right]
  \]
\end{example}

\section{行基本変形による計算}


定理\ref{thm:formula}の直交基底への変換行列 $M$ は $n \times
2n$ 行列 $\left[ G | E\right]$ に行基本変形「ある行に別の行のスカラー倍
を加える」を有限回施すことで計算することでもできる.例えば,
例\ref{exmp:3-dim} の $G$ に対しては
\[
  \begin{aligned}
    \left[ G \ | \ E_3\right]  & =\left[
      \begin{array}{rrr|rrr}
        10 & 4 & 3 & 1 & 0 & 0\\
        4 & 2 & 1 & 0 & 1 & 0\\
        3 & 1 & 5 & 0 & 0 & 1
      \end{array}
    \right]   \to  \left[
      \begin{array}{rrr|rrr}
        10 & 4 & 3 & 1 & 0 & 0\\
        0 & \frac{2}{5} & -\frac{1}{5} & -\frac{2}{5} & 1 & 0\\
        0 & -\frac{1}{5} & \frac{41}{10} & -\frac{3}{10} & 0 & 1
      \end{array}
    \right]\\
    &\to \left[
      \begin{array}{rrr|rrr}
        10 & 4 & 3 & 1 & 0 & 0\\
        0 & \frac{2}{5} & -\frac{1}{5} & -\frac{2}{5} & 1 & 0\\
        0 & 0 & 4 & -\frac{1}{2} & \frac{1}{2} & 1
      \end{array}
\right]
  \end{aligned}
\]
と左 $3$ 列が上三角行列になるように変形できる.このとき,右 $3$ 列の転置
行列が変換行列 $M$ である.つまり,
\[
  M = \left[
    \begin{array}{rrr}
      1 & -\frac{2}{5} & -\frac{1}{2}\\
      0 & 1 & \frac{1}{2}\\
      0 & 0 & 1
    \end{array}
  \right]
\]
である.さらに,変形後の左 $3$ 列の上三角行列の対角成分には直交基
底 $\{\bm{w}_1, \bm{w}_2, \bm{w}_3\}$ の自己内積が並ぶ.つまり,
\[
  (\bm{w}_1, \bm{w}_1) = 10, \quad (\bm{w}_2, \bm{w}_2) = \frac{2}{5}, \quad
  (\bm{w}_3, \bm{w}_3) = 4
\]
である.これは\cite{gauss}の結果をほんの少しだけ拡張している.
この計算過程の正当性を説明する.\\


$G=\left[ g_{ij}\right]$ を実内積空間 $V$
の基底$\{\bm{v}_1, \ldots, \bm{v}_n\}$ の Gram 行列とする.$G$ は正定値
対称行列なので,行基本変形の「 $i$ 行の $c$ 倍を $j$ 行に加え
る $(i<j)$」のみを有限回施すことで上三角行列 $U$ へ変形できる(補
題\ref{lem:rowadd}として後述).そのような行基本変形に対応する基本行列
は,いずれも全ての対角成分が $1$の下三角行列なので,それら全ての積 $L$
も対角成分が全て $1$ の下三角行列であり,$U=LG$ となる.従っ
て,$\left[ G \ | \ E \right]$ に上記のような行基本変形を施せば
\[
  \left[ G \ | \ E\right] \to L\left[ G \ | \ E \right] = \left[ LG \
    |\ L \right] = \left[ U \ | \ L\right]
\]
となる.ここで,$M={}^{t}L$ として
\[
  \left( \bm{w}_1, \bm{w}_2, \ldots \bm{w}_n \right) = \left(
    \bm{v}_1, \bm{v}_2, \ldots, \bm{v}_n\right) M
\]
と基底を変換し,$\{\bm{w}_1, \ldots, \bm{w}_n\}$ の Gram 行列を $H$ と
すると $H$ は対称行列であり,
\[
  H=\left[ (\bm{w}_i, \bm{w}_j)\right] = {}^{t}M G M = L G M = U M
\]
より $H$ は上三角行列でもある.従って,$H$ は対角行列なの
で $\{\bm{w}_1, \ldots, \bm{w}_n\}$ は $V$ の直交基底である.また,$M$
の対角成分は全て $1$ なので,$H$ と $U$ の対角成分は等しい.つま
り,$(\bm{w}_i, \bm{w}_i)$ は $U$ の $(i,i)$ 成分に等しい.さらに,$M$
の対角成分が全て $1$ であることはこの $\{\bm{w}_1, \ldots,
\bm{w}_n\}$ が定理\ref{thm:formula} の直交基底に等しいことを表してい
る.

\newpage

先の説明で省いた部分を補題\ref{lem:rowadd}として示しておく.

\begin{lemma}\label{lem:rowadd}
  正定値対称行列 $G$ は,$i <j$ である行基本変形
  「 $i$ 行の $c$ 倍を $j$ 行に加える」のみを有限回施すことで,上三角行
  列へと変形できる.
\end{lemma}

\begin{proof}

  $G=\left[ g_{ij}\right]$ とし,その次数 $n$ に関する帰納法によ
  る.$n=1$ のときは明らか.$n \geq 2$ とする.$G$ の正定値性か
  ら $g_{11} >0$ なので,$G$ に行基本変形「$1$ 行の $-g_{i1}/g_{11}$ 倍
  を $i$ 行に加える」を $i=2, \ldots, n$として順次施して
  \[
   G \to L G=\left[
      \begin{array}{cc}
        g_{11} & \bm{\ast}\\
        \bm{0} & G'
      \end{array}
    \right]
  \]
  と変形できる.ここで,$L$ は上記の行基本変形に対応する基本行列たち
  の積であり,下三角行列であ
  る.$H:=L G~{}^{t}L$ は対称行列なので
  \[
    L = \left[
      \begin{array}{cc}
        1 & \bm{0}\\
        \bm{\ast} & L'
      \end{array}
    \right]
  \]
  とおくと,
  \[
    H = \left[
      \begin{array}{cc}
        g_{11}  & \ast \\
        \bm{0} & G'
      \end{array}
    \right] \left[
      \begin{array}{cc}
        1 & \ast\\
        \bm{0} & {}^{t}L'
      \end{array}
    \right] = \left[
      \begin{array}{cc}
        g_{11} & \ast\\
        \bm{0} & G'~{}^{t}L'
      \end{array}
    \right] = \left[
      \begin{array}{cc}
        g_{11} & \bm{0}\\
        \bm{0} & H'
      \end{array}
    \right]
  \]
  となり,$H':=G'~{}^{t}L'$ も対称行列である.さらに,任意の $\bm{x} \in
  \mathbb{R}^n$ に対して ${}^{t}L \bm{x} \in \mathbb{R}^n$ なの
  で,$G$ の正定値性から
  \[
    {}^{t}\bm{x} H \bm{x} = 
    {}^{t}\bm{x}\left( L G~ {}^{t}L \right)\bm{x} = {}^{t}\left(
      {}^{t}L \bm{x}\right) G \left({}^{t}L \bm{x}\right) >0
  \]
  より,$H$ は正定値である.従って,$H'$ も正定値対称行列なので,帰納法
  の仮定から行基本変形「$i$ 行の $c$ 倍を $j$ 行に加える $(i<j)$」のみ
  を有限回施して上三角行列 $U_1'$ へと変形でき,そのような行基本変形に
  対応する基本行列たちの積を $L_1'$ とすれば $U'=L_1'H' =
  L_1'G'~{}^{t}L'$ である.${}^{t}L'$ は対角成分が全て $1$ の上三角行列
  なので正則で,その逆行列 $\left({}^{t}L'\right)^{-1}$ は上三角行列だ
  から
  \[
    L_1' G' = U'\left( {}^{t}L'\right)^{-1}
  \]
  は上三角行列である.よって,
  \[
    L_1 = \left[
      \begin{array}{cc}
        1 & \bm{0}\\
        \bm{0} & L_1'
      \end{array}
    \right]
  \]
  とおけば
  \[
    L_1 L G = \left[
      \begin{array}{cc}
        1 & \bm{0}\\
        \bm{0} & L_1'
      \end{array}
    \right] \left[
      \begin{array}{cc}
        g_{11} & \bm{\ast} \\
        \bm{0} & G'
      \end{array}
    \right] = \left[
      \begin{array}{cc}
        g_{11} & \bm{\ast}\\
        \bm{0} & L_1' G'
      \end{array}
    \right]
  \]
  より,$L_1 L G$ は上三角行列である.
\end{proof}

\newpage

\begin{example}\renewcommand{\arraystretch}{1.2}
  
  $2$ 次以下の多項式のなすベクトル空間 $\mathbb{R}[x]_2$ に
  \[
    (f,g) = \int_{0}^{1} f(x) g(x) dx \quad \left( f, g \in \mathbb{R}[x]_2\right)
  \]
  として内積を定める.この内積空間の直交基底を求めるために,基
  底 $\{1, x, x^2\}$ を直交化する.まず,$i,j=0,1,2$ に対して
  \[
    (x^i, x^j) = \int_{0}^{1} x^{i+j} \ dx = \left[
      \frac{1}{i+j+1}x^{i+j+1} \right]_{0}^{1}= \frac{1}{i+j+1}
  \]
  より,基底 $\{1, x, x^2\}$ の Gram 行列は
    \[
      G= \left[ \frac{1}{i+j-1} \right] =\left[
        \begin{array}{ccc}
          1 & \frac{1}{2} & \frac{1}{3}\\
          \frac{1}{2} & \frac{1}{3} & \frac{1}{4}\\
          \frac{1}{3} & \frac{1}{4} & \frac{1}{5}
        \end{array}
      \right]
    \]
  である.$\left[ G \ | \ E_3\right]$ を変形する.
    \[
      \left[
        \begin{array}{ccc|ccc}
          1 & \frac{1}{2} & \frac{1}{3}& 1 & 0 & 0\\
          \frac{1}{2} & \frac{1}{3} & \frac{1}{4} & 0 & 1 & 0\\
          \frac{1}{3} & \frac{1}{4} & \frac{1}{5} & 0 & 0 & 1
        \end{array}
      \right] \to \left[
        \begin{array}{ccc|rcc}
          1 & \frac{1}{2} & \frac{1}{3} & 1 & 0 & 0\\
          0 & \frac{1}{12} & \frac{1}{12} & -\frac{1}{2} & 1 & 0\\
          0 & \frac{1}{12} & \frac{4}{45} & -\frac{1}{3} & 0 & 1
        \end{array}
      \right] \to \left[
        \begin{array}{ccc|rrc}
          1 & \frac{1}{2} & \frac{1}{3} & 1 & 0 & 0\\
          0 & \frac{1}{12} & \frac{1}{12} & -\frac{1}{2} & 1 & 0\\
          0 & 0 & \frac{1}{180} & \frac{1}{6} & -1 & 1
        \end{array}
      \right]
    \]
  これより
  \[
    (f_1, f_2, f_3) = \left( 1, x, x^2\right) \left[
      \begin{array}{rrr}
        1 & -\frac{1}{2} & \frac{1}{6}\\
        0 & 1 & -1\\
        0 & 0 & 1
      \end{array}
    \right] = \left( 1, \quad -\frac{1}{2} + x, \quad \frac{1}{6} -x+x^2\right)
  \]
  とすれば,$\{f_1, f_2, f_3\}$ は直交基底で,各多項式の自己内積は以下の通りである.
  \[
    (f_1, f_1) = 1, \quad (f_2, f_2) = \frac{1}{12}, \quad (f_3,f_3) = \frac{1}{180}
  \]


 \renewcommand{\arraystretch}{1}
\end{example}



\begin{thebibliography}{9}

\bibitem{miyake} 三宅敏恒,「入門線形代数」(1991),培風館.
  
\bibitem{gauss} L.~Pursell and S.~Y.~Trimble, ``Gram-Schmidt
  Orthogonalization by Gauss Elimination'', The American Mathematical
  Monthly, \textbf{98}, No.6 (1991), p.544-549.
\end{thebibliography}

\end{document}
