\documentclass[11pt, uplatex, dvipdfmx]{jsarticle} 
\usepackage{amsmath,amsthm,amssymb,amsfonts,fancyhdr, enumerate, braket, setspace, graphicx, bm, multirow, bigdelim}
\newcommand{\ds}{\displaystyle}

\theoremstyle{definition}
\newtheorem{theorem}{定理}
\newtheorem*{theorem*}{定理}
\newtheorem*{definition}{定義}
\newtheorem*{remark}{注意}
\newtheorem*{example}{例}
\renewcommand{\proofname}{\textbf{証明}}


\renewcommand{\arraystretch}{1.3}

\pagestyle{fancy}

\rhead{}
\lhead{}

\chead{Gram-Schmidt の正規直交化法に関する明示公式}

\begin{document}

$\{\bm{v}_1, \ldots, \bm{v}_n\}$ を実内積空間 $V$ の基底とし,その Gram
行列を
$G = \left[ g_{ij} \right]= \left[ \left(\bm{v}_i, \bm{v}_j
  \right)\right]$ とする.$G$ は正定値対称行列である.$V$ の基
底 $\{\bm{v}_1, \ldots, \bm{v}_n\}$ に Gram-Schmidt の正規直交化法を適
用して得られる $V$ の正規直交基底 $\{\bm{u}_1, \ldots, \bm{u}_n\}$ の明示公式を記す.

$G$ の $k$ 次主対角行列を $G^{(k)}$ とし,その $(i,j)$ 余因子を $\tilde{g}^{(k)}_{ij}$ とする.つまり,
\[
  G^{(k)} := \left[
    \begin{array}{ccc}
      g_{11} & \cdots & g_{1k}\\
      \vdots & \ddots & \vdots\\
      g_{k1} & \cdots & g_{kk}
    \end{array}
  \right], \quad \tilde{g}^{(k)}_{ij} := (-1)^{i+j} \left|G^{(k)}_{ij}\right|
\]
とする.ここで,$G^{(k)}_{ij}$ は $G^{(k)}$ から $i$ 行と $j$ 列を取り除いた $k-1$ 次行列を表す.

\begin{theorem*}[Gram-Schmidt の正規直交化法の明示公式]
  上の設定のもとで
  \[
    \bm{w}_1= \bm{v}_1, \quad \bm{w}_k =
    \frac{1}{\left|G^{(k-1)}\right|} \sum_{i=1}^{k}
    \tilde{g}^{(k)}_{ik} \bm{v}_i = \frac{1}{\left|G^{(k-1)}\right|}\left|
      \begin{array}{ccc}
        g_{11} & \cdots & g_{1k}\\
        \vdots & \ddots & \vdots\\
        g_{k-1,k} & \cdots & g_{k-1, k}\\
        \bm{v}_1 & \cdots & \bm{v}_k
      \end{array}
    \right| \; ( 2 \leq k \leq n)
  \]
  とすれば,$\bm{w}_1, \ldots, \bm{w}_n$ は互いに直交する.さらに,
  \[
    (\bm{w}_1, \bm{w}_1) = g_{11}, \quad (\bm{w}_k, \bm{w}_k) =
    \frac{\left| G^{(k)}\right|}{\left|G^{(k-1)}\right|} \; (2 \leq k \leq n)
  \]
  となる.これより
  \[
    \bm{u}_1 = \frac{\bm{w}_1}{\sqrt{g_{11}}}, \quad \bm{u}_k =
    \sqrt{\frac{\left|G^{(k-1)}\right|}{\left|G^{(k)}\right|}}~ \bm{w}_k \; (2 \leq k \leq n)
  \]
  とすれば,$\{\bm{u}_1, \ldots, \bm{u}_n\}$ は $V$ の正規直交基底である.つまり,
  \[
    \mu_{ij} = \frac{\tilde{g}^{(j)}_{ij}}{\tilde{g}^{(j)}_{jj}}
    = \frac{\tilde{g}^{(j)}_{ij}}{\left|G^{(j-1)}\right|}, \quad
    \nu_{i} = \sqrt{\frac{\left|G^{(i-1)}\right|}{\left|G^{(i)}\right|}}
  \]
  とおいて
  \[
    M = \left[
      \begin{array}{ccccc}
        1 & \mu_{12} & \mu_{13} & \cdots & \mu_{1n}\\
         & 1 & \mu_{23} & \cdots & \mu_{2n}\\
         &  & 1 & \cdots & \mu_{3n}\\
         &  &  & \ddots & \vdots\\
        O &  &  &  & 1
      \end{array}
    \right], \quad N= \left[
      \begin{array}{ccccc}
        \sqrt{1/g_{11}} & & & & O\\
                        & \nu_{2} & & & \\
                        & & \nu_{3} & &\\
                        & && \ddots & \\
        O & & &&  \nu_{n}
      \end{array}
    \right]
  \]
  とすれば,以下が成り立つ.
  \[
    \begin{aligned}
      \left(\bm{u}_1, \bm{u}_2, \cdots, \bm{u}_n\right) = \left(\bm{w}_1, \bm{w}_2, \cdots, \bm{w}_n\right) N=
    \left( \bm{v}_1 , \bm{v}_2, \cdots, \bm{v}_n\right)MN
  \end{aligned}
  \]
\end{theorem*}

\begin{proof}
  Gram-Shcmidt の正規直交化法により
  \[
    \bm{w}_1 = \bm{v}_1, \quad \bm{w}_k = \bm{v}_k - \sum_{i=1}^{k-1}
    \frac{(\bm{v}_k, \bm{w}_{i})}{(\bm{w}_i,\bm{w}_i)}\bm{w}_i \; (2 \leq k \leq n)
  \]
  \[
    \bm{u}_i = \frac{\bm{w}_i}{\|\bm{w}_i\|} \; (1 \leq i \leq n)
  \]
  とすれば,$\{\bm{u}_1, \ldots, \bm{u}_n\}$ は $V$ の正規直交基底となる.このとき,各 $k=2,\ldots, n$ に対して
  \[
    \langle \bm{w}_1, \ldots, \bm{w}_{k-1}\rangle =\langle \bm{v}_1,
    \ldots, \bm{v}_{k-1}\rangle
  \]
  なので,
  \[
    \bm{w}_k = \sum_{i=1}^{k-1}\mu_{i k} \bm{v}_i + \bm{v}_k = \left(\bm{v}_1, \ldots, \bm{v}_{k}\right) \left[
      \begin{array}{c}
        \mu_{1k}\\
        \vdots\\
        \mu_{k-1,k}\\
        1
      \end{array}
    \right]\; (\mu_{ik} \in \mathbb{R})
  \]
  と書ける(添字の片方または両方が2文字以上になるときはコンマで区切る).つ
  まり,
  \[
    M = \left[
      \begin{array}{ccccc}
        1 & \mu_{12} & \mu_{13} & \cdots & \mu_{1n}\\
          & 1 & \mu_{23} & \cdots & \mu_{2n}\\
          &  & 1 & \cdots & \mu_{3n}\\
          &  &  & \ddots & \vdots\\
        O &  &  &  & 1
      \end{array}
    \right]
  \]
  によって $(\bm{w}_1, \ldots, \bm{w}_n) = (\bm{v}_1, \ldots,
  \bm{v}_n) M$ と書ける.この各 $\mu_{ij}$ を決定する.
  
  各 $k=2,\ldots, n$ に対して
  $\bm{w}_k \in \langle \bm{w}_1, \ldots, \bm{w}_{k-1}\rangle^{\perp} =
  \langle\bm{v}_1, \ldots, \bm{v}_{k-1}\rangle^{\perp} $ なので,
  \[
    (\bm{v}_1, \bm{w}_k) = (\bm{v}_2, \bm{w}_k) = \cdots = (\bm{v}_{k-1}, \bm{w}_k) = 0
  \]
  である.従って,$\{\bm{e}_{1}, \ldots,
  \bm{e}_k\}$ を $\mathbb{R}^k$ の標準基底とすれば,各 $i=1,
  \ldots, k-1$ に対して
  \[
    0  =(\bm{v}_i, \bm{w}_k) = {}^{t}\bm{e}_{i}
    G^{(k)} \left[
      \begin{array}{c}
        \mu_{1k}\\
        \vdots\\
            1
      \end{array}
    \right]
    = \left[
      \begin{array}{ccc}
        g_{i1} & \cdots & g_{ik}
      \end{array}
    \right] \left[
      \begin{array}{c}
        \mu_{1k}\\
        \vdots \\
        1
      \end{array}
    \right] 
  \]
  である.従って,$\bm{g}_i = {}^{t}\left[
    \begin{array}{ccc}
      g_{i1} & \cdots & g_{ik}
    \end{array}
  \right], \; \bm{\mu}_k = {}^{t}\left[
    \begin{array}{cccc}
      \mu_{1k} & \cdots & \mu_{k-1,k} & 1
    \end{array}
  \right]$ とすれば,$\bm{\mu}_k$ は $\mathbb{R}^k$ の標準内積のもと
  で $\bm{g}_1, \ldots, \bm{g}_{k-1}$ に直交するので以下のように書ける.
  \[
    \bm{\mu}_k = m_k \left|
      \begin{array}{ccc}
        g_{11} & \cdots & g_{1k}\\
        \vdots & \ddots & \vdots\\
        g_{k-1,1} & \cdots & g_{k-1,k}\\
        \bm{e}_1 & \cdots & \bm{e}_k
      \end{array}
    \right| = m_k \sum_{j=1}^{k} \tilde{g}^{(k)}_{jk} \bm{e}_j
    = m_k \left[
      \begin{array}{c}
        \tilde{g}^{(k)}_{1k}\\
        \vdots\\
        \tilde{g}^{(k)}_{kk}
      \end{array}
    \right]\quad ( m_k \in \mathbb{R})
  \]
  $\bm{\mu}_k$ の $\bm{e}_k$ 成分は $1$ なので
  $m_k = 1/\tilde{g}^{(k)}_{kk} = 1/\left|G^{(k-1)}\right|$ である.以
  上から,$k=2, \ldots, n$ に対して
  \[
    \bm{w}_k = (\bm{v}_1, \ldots, \bm{v}_k) \bm{\mu}_k
    = \frac{1}{\left| G^{(k-1)} \right|} \sum_{i=1}^{k} \tilde{g}^{(k)}_{ik} \bm{v}_k
    = \frac{1}{\left|G^{(k-1)} \right|}\left|
      \begin{array}{ccc}
        g_{11} & \cdots & g_{1k}\\
        \vdots & \ddots & \vdots\\
        g_{k-1,1} & \cdots & g_{k-1,k}\\
        \bm{v}_1 & \cdots & \bm{v}_k
      \end{array}
    \right|
  \]
  を得る.さらに,$\|\bm{w}_1\| = \sqrt{g_{11}}$ であり,$k=2, \ldots, n$ に対しては
  \[
    \begin{aligned}
      (\bm{w}_k, \bm{w}_k) &= {}^{t}\bm{\mu}_k G^{(k)} \bm{\mu}_k = {}^{t} \bm{\mu}_k \left[
        \begin{array}{c}
          {}^{t}\bm{e}_1 G^{(k)}\bm{\mu}_k\\
          \vdots \\
          {}^{t} \bm{e}_{k} G^{(k)}\bm{\mu}_k\\
        \end{array}
      \right] = \left[
        \begin{array}{cccc}
          \mu_{1k} & \cdots & \mu_{k-1,k} & 1
        \end{array}
      \right] \left[
        \begin{array}{c}
          0\\
          \vdots\\
          0\\
          {}^{t}\bm{e}_k G^{(k)}\bm{\mu}_k
        \end{array}
      \right]\\
      &= {}^{t} \bm{e}_k G^{(k)}\bm{\mu}_k = \sum_{i=1}^{k-1} g_{ki} \mu_{ik} + g_{kk}
      = \frac{1}{\left| G^{(k-1)}\right|}\sum_{i=1}^{k}g_{ik} ~ \tilde{g}^{(k)}_{ik}
      = \frac{\left| G^{(k)}\right|}{\left|G^{(k-1)}\right|}
    \end{aligned}
  \]
  なので,$\nu_k = 1/\|\bm{w}_k\| = \sqrt{\left|G^{(k-1)}\right|/\left|G^{(k)}\right|}$ として
  \[
    N = \left[
      \begin{array}{cccc}
        1/\sqrt{g_{11}} & & & O\\
                       & \nu_2 & & \\
                       & & \ddots & \\
        O & & & \nu_n
      \end{array}
    \right]
  \]
  によって $(\bm{u}_1, \ldots, \bm{u}_m) = (\bm{w}_1, \ldots, \bm{w}_n) N$ と書ける.
\end{proof}


\begin{remark}
  各 $\bm{w}_k \; (k \geq 2)$ は $\bm{v}_1, \ldots, \bm{v}_{k-1}$ に直
  交するので,
  \[
    \begin{aligned}
      \bm{0}_{k-1} &= \left[
        \begin{array}{c}
          (\bm{v}_1, \bm{w}_k)\\
          \vdots\\
          (\bm{v}_{k-1}, \bm{w}_k)
        \end{array}
      \right] = \left[
        \begin{array}{c}
          {}^{t}\bm{e}_1 G^{(k)} \bm{\mu_k}\\
          \vdots\\
          {}^{t} \bm{e}_{k-1} G^{(k)} \bm{\mu}_k
        \end{array}
      \right] = \left[
        \begin{array}{c}
          {}^{t}\bm{e}_1 G^{(k)}\\
          \vdots\\
          {}^{t}\bm{e}_{k-1}G^{(k)}
        \end{array}
      \right] \bm{\mu}_k\\
      &=\left[
        \begin{array}{cc}
          & g_{1k}\\
          G^{(k-1)} & \vdots \\
           & g_{k-1,k}
        \end{array}
      \right] \left[
        \begin{array}{c}
          \mu_{1k}\\
          \vdots\\
          \mu_{k-1,k}\\
          1
        \end{array}
      \right]
      = G^{(k-1)} \left[
        \begin{array}{c}
          \mu_{1k}\\
          \vdots \\
          \mu_{k-1,k}
        \end{array}
        \right] + \left[
          \begin{array}{c}
            g_{1k}\\
            \vdots\\
            g_{k-1,k}
          \end{array}
        \right]
    \end{aligned}
  \]
  となる.つまり,$\left[
    \begin{array}{c}
      \mu_{1k}\\
      \vdots\\
      \mu_{k-1,k}
    \end{array}
    \right]$ は連立1次方程式 $G^{(k-1)} \bm{x} = -\left[
      \begin{array}{c}
        g_{1k}\\
        \vdots\\
        g_{k-1,k}
      \end{array}
\right]$ の解でもある.
\end{remark}

\newpage

\chead{行基本変形による直交基底の計算法}

実内積空間の Gram 行列 $G$ から行基本変形を用いて直交基底への変換行列 $M$ を計算できる.

$G$ は正定値対称行列なので,$n \times 2n$ 行列 $\left[ G \ | \
  E_n\right]$ に「ある行に別の行のスカラー倍を加える」という行基本変形
だけを有限回施して左 $n$ 列が上三角行列 $U$ となるように変形できる. こ
のとき,$\left[ G \ | \ E_n\right] \to \left[ U \ | \ {}^{t}M \right]$
となる.さらに,$U$ の対角成分には直交基底の2乗ノルムが並ぶ.

\begin{example}
  $4$ 次元実内積空間 $V$ の基底 $\{\bm{v}_1, \bm{v}_2, \bm{v}_3,
  \bm{v}_4\}$ の Gram 行列 $G=\left[ (\bm{v}_i, \bm{v}_j)\right]$ が
  \[
    G =\left[
      \begin{array}{rrrr}
        27 & 8 & 5 & 14\\
        8 & 3 & 2 & 4\\
        5 & 2 & 13 & 5\\
        14 & 4 & 5 & 11
      \end{array}
    \right]
  \]
  であるとする.このとき,直交基底 $\{\bm{w}_1, \ldots, \bm{w}_n\}$ を
  計算しよう.$4 \times 8$ 行列 $\left[ G \ | \
    E_4\right]$ に「ある行に別の行のスカラー倍を加える」という行基本変
  形だけを適宜施して
  \[
    \left[
      \begin{array}{rrrr|rrrr}
        27 & 8 & 5 & 14 & 1 & 0 & 0 & 0\\
        8 & 3 & 2 & 4 & 0 & 1 & 0 & 0 \\
        5 & 2 & 13 & 5 & 0 & 0 & 1 & 0\\
        14 & 4 & 5 & 11 & 0 & 0 & 0 & 1
      \end{array}
    \right] \longrightarrow \left[
      \begin{array}{rrrr|rrrr}
        27 & 8 & 5 & 14 & 1 & 0 & 0 & 0\\
        0 & \frac{17}{27} & \frac{14}{27} & -\frac{4}{27} & -\frac{8}{27} & 1 & 0 & 0\\
        0 & 0 & \frac{198}{17} & \frac{43}{17} & \frac{1}{17} & -\frac{14}{17} & 1 & 0\\
        0 & 0 & 0 & \frac{625}{198} & -\frac{119}{198} & \frac{41}{99} & -\frac{43}{198} & 1
      \end{array}
    \right]
  \]
  とできる.これより
  \[
    U = \left[
      \begin{array}{rrrr}
         27 & 8 & 5 & 14 \\
        0 & \frac{17}{27} & \frac{14}{27} & -\frac{4}{27}\\
        0 & 0 & \frac{198}{17} & \frac{43}{17}\\
        0 & 0 & 0 & \frac{625}{198}
      \end{array}
    \right], \quad M = \left[
      \begin{array}{rrrr}
        1 & -\frac{8}{27} & \frac{1}{17} & -\frac{119}{198}\\
        0 & 1 & -\frac{14}{17} & \frac{41}{99}\\
        0 & 0 & 1 & -\frac{43}{198}\\
        0 & 0 & 0 & 1
      \end{array}
    \right]
  \]
  である.従って,
  \[
    (\bm{w}_1, \bm{w}_2, \bm{w}_3, \bm{w}_4) = (\bm{v}_1, \bm{v}_2, \bm{v}_3, \bm{v}_4) M
  \]
  とすれば,$\{\bm{w}_1, \bm{w}_2, \bm{w}_3, \bm{w}_4\}$ は $V$ の直交
  基底である.さらに,$U$ の対角成分がそれぞれの $2$ 乗ノルムである.つまり,
  \[
    (\bm{w}_1, \bm{w}_1) = 27, \quad (\bm{w}_2, \bm{w}_2) = \frac{17}{27}, \quad
    (\bm{w}_3, \bm{w}_3) = \frac{198}{17}, \quad (\bm{w}_4, \bm{w}_4) = \frac{625}{198}
  \]
  なので
  \[
    \bm{u}_1 = \sqrt{\frac{1}{27}}\bm{w}_1, \quad \bm{u}_2 = \sqrt{\frac{27}{17}}\bm{w}_2, \quad
    \bm{u}_3 = \sqrt{\frac{17}{198}}\bm{w}_3, \quad \bm{u}_4=\sqrt{\frac{198}{625}} \bm{w}_4
  \]
  とすれば,$\{\bm{u}_1, \bm{u}_2, \bm{u}_3, \bm{u}_4\}$ は $V$ の正規直交基底である.
\end{example}

\begin{thebibliography}{9}
\bibitem{gauss} L.~Pursell and S.~Y.~Trimble, ``Gram-Schmidt
  Orthogonalization by Gauss Elimination'', The American Mathematical
  Monthly, \textbf{98}, No.6 (1991), p.544-549.
\end{thebibliography}

\end{document}
